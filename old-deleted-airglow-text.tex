thus shifts terrestrial airglow lines away from their rest wavelengths. To properly ascertain whether or not any of our unexplained candidates matched the expanded set of airglow lines, we had to undo the barycentric shift and put them in the Earth's reference frame. We utilized the barycorr tool developed in \cite{2014PASP..126..838W} to determine the direction and magnitude of the barycentric shift which would have been added by the reduction pipeline, then applied it in reverse. We then compared each Earth-corrected wavelength to all nearby airglow wavelengths. As a validation of this technique, we first tested it on a sample set of the prominent airglow lines normally filtered out by our search algorithm. 

As shown in \ref{table:airglowwavelengthcomparison}, none the unexplained spikes matched nearby airglow lines once the barycentric correction was undone/the wavelengths placed into  Earth's reference frame. This  indicates that airglow is not a viable explanation for any of these candidates. For additional context, in \ref{table:airglowvelocitycomparison} we can see a comparison between a ``predicted'' doppler velocity treating the airglow $\lambda$ as the rest wavelength, and the actual barycentric shift velocity given by barycorr. After applying the shift to the rest wavelengths, none of them match the candidate wavelengths. Furthermore, the barycentric shift predicted by the observed wavelength as compared to the rest wavelenghts of the airglow lines diverge considerably In every case, there was a substantial difference, sometimes on the order of tens of kilometers per second. In several cases, they also show opposite directionality. This further indicates that nearby faint airglow lines are not a viable explanation for any of these candidates.

\begin{table}
\begin{center}
    \begin{tabular}{|c|c|c|c|c|c|c|}
    \hline 
    Star & Observation Date/Time  & $v_{\mbox{bary}}$ & $\lambda_{\mbox{bary}}$ & $\lambda_{\mbox{obs}}$ & $\lambda_{\mbox{airglow}}$ & $\Delta v$ (obs - airglow) \\
    & & (km/s) & (\AA) & (\AA) & (\AA) & (km/s) \\
    \hline
    CD-312415 & 2018-02-11, 01:10:40.597 & -15.983 & 3931.24 & 3931.03 & 
    3931.0825 & 12.011 \\
    GJ317 &  2018-04-10, 03:25:08.275 &  -17.533 & 4662.77 & 4662.49 & 
    4662.0542 & 46.030 \\
    GJ317 &  2018-04-10, 03:25:08.275 &  -17.533 & 4662.77 & 4662.49 & 
    4662.2324 & 34.569 \\
    %HD96673 & 2006-01-28, 07:17:55.086 &  15.715 & 3931.31 & 3931.51 & 
    %3931.0825 & 17.303 \\
    HIP87607 & 2012-10-22, 23:39:09.238 & -23.949 & 3784.09 & 3783.78 &
    3784.8796 & -62.544 \\
    GJ4291 & 2005-07-20, 04:08:04.468 & 20.089 &  3841.74 & 3841.997 &
    3841.3704 & 3.277 \\
    GJ4291 & 2005-07-20, 04:08:04.468 &  20.089 &  3841.74 & 3841.997 &
    3841.6980  & -32.819 \\
    GJ4291 & 2005-07-20, 04:08:04.468 &  20.089 &  3841.74 & 3841.997 &
    3842.1606  & 20.089 \\
    \hline
    \end{tabular}
\end{center}    
\caption{Comparison of candidate spectral signatures with nearby terrestrial airglow lines.  $v_{\mbox{bary}}$:  Relative velocity between solar barycenter and observatory, in direction of star.  $\lambda_{\mbox{bary}}$: wavelength of candidate spikes, in barycentric frame, as reported by HARPS spectral files.  $\lambda_{\mbox{obs}}$: wavelength of candidate spikes in observatory frame, calculated using  $v_{\mbox{bary}}$. 
 $\lambda_{\mbox{airglow}}$: wavelength of nearby airglow line. $\Delta v$: Velocity of airglow source relative to observaotry that would be needed to explain candidate as an airglow line.  The large values for $\Delta v$ indicate that the candidates cannot be explained by airglow. }
\end{table}

\begin{table}
\begin{center}
\begin{tabular}{|c|c|c|c|c|} 
 \hline
 Star & Observation Date/Time & $\lambda$ & $\lambda$ (Earth Reference Frame) & $\lambda$ (Airglow) & Velocity (km/s) \\ 
 \hline
 CD-312415 & 2018-02-11, 01:10:40.597 & 3931.24 \AA & 3931.03 \AA & 3931.0825 \AA \\
 \hline
 GJ317 &  2018-04-10, 03:25:08.275 &  4662.77 \AA  & 4662.49 \AA & 4662.0542 \AA, 4662.2324 \AA \\ 
 \hline  
 HD96673 & 2006-01-28, 07:17:55.086 & 3931.31 \AA & 3931.51 \AA & 3931.0825 \AA \\
 \hline
 HIP87607 & 2012-10-22, 23:39:09.238 & 3784.09 \AA & 3783.78 \AA & 3784.8796 \AA \\
 \hline
 GJ4291 & 2005-07-20, 04:08:04.468 & 3841.74 \AA & 3841.997 \AA & 3841.3704 \AA, 3841.6980 \AA, 3842.1606 \AA \\
 \hline
 \end{tabular}
\end{center}
\caption{.}
\label{table:airglowwavelengthcomparison}
\end{table}
%NOTE: I wanted this to be one table instead of two, but it had too many cells to be contained in the page. Jason, any way to consolidate these two?
\begin{table}
\begin{center}
\begin{tabular}{|c|c|c|c|c|} 
 \hline
Star & Observation Date/Time & $\lambda$ & Predicted Doppler Velocities & Barrycorr Velocity \\ 
 \hline
 CD-312415 & 2018-02-11, 01:10:40.597  &  3931.24 \AA & 12.011 km/s & -15.983 km/s \\
 \hline
 GJ317 & 2018-04-10, 03:25:08.275 &  4662.77 \AA & 46.030 km/s, 34.569 km/s & -17.533 km/s\\ 
 \hline
 HD96673 & 2006-01-28, 07:17:55.086 & 3931.31 & 17.303 km/s & 15.715 km/s \\
 \hline
 HIP87607 & 2012-10-22, 23:39:09.238 & 3784.09 \AA & -62.544 km/s & -23.949 km/s \\
 \hline
 GJ4291 & 2005-07-20, 04:08:04.468 & 3841.74 \AA & 28.845 (km/s), 3.277 (km/s), -32.819 (km/s) &  20.089 (km/s)\\
 \hline
 \end{tabular}
\end{center}
\caption{.}
\label{table:airglowvelocitycomparison}
\end{table}
